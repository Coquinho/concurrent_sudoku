\documentclass[a4paper, 12pt]{article}

\usepackage[brazilian]{babel}
\usepackage[utf8]{inputenc}
\usepackage{indentfirst}
\usepackage{fontspec}
\usepackage[urlcolor=blue, linkcolor=black, colorlinks=true]{hyperref}
\usepackage{fancyvrb}
\usepackage{geometry}

\geometry{
    a4paper,
    left = 20mm,
    right = 20mm,
    top = 20mm,
}

\begin{document}
\begin{titlepage}
    \begin{center}
        \huge{Universidade Federal de Santa Catarina}

        \vspace{10pt}

        \vspace{85pt}

        \textbf{\LARGE{Sistemas Operacionais I}}
        \large{\\Trabalho 1}
        \vspace{160pt}

    \end{center}

    \begin{flushleft}
        \begin{tabbing}
            Alunos\qquad\qquad\=
            Filipe Oliveira de Borba\\
            \>Guilherme Christopher Michaelsen Cardoso\\
            \>Lucas Cavalcante Sousa\\\\
            Professor\> Márcio Bastos Castro \\
        \end{tabbing}
    \end{flushleft}

    \begin{center}
        \vspace{\fill}
        Florianópolis, \today
    \end{center}
\end{titlepage}

%%%%%%%%%%%%%%%%%%%%%%%%%%%%%%%%%%%%%%%%%%%%%%%%%%%%%%%%%%%%%%%%%%%%%%%%%%%%%%%%

\newpage
\tableofcontents
\thispagestyle{empty}

%%%%%%%%%%%%%%%%%%%%%%%%%%%%%%%%%%%%%%%%%%%%%%%%%%%%%%%%%%%%%%%%%%%%%%%%%%%%%%%%

\newpage
\pagenumbering{arabic}

\section{Sudoku}
O Sudoku é um quebra cabeça de disposição de números. Um Sudoko 9x9, por exemplo,
está correto quando os números de 1 à 9 estão preenchendo todos os espaços vazios
de forma a não haver nenhuma repetição nas linhas, colunas nem regiões.

\section{Verificando a Solução de um Sudoku}

Verificar a solução de um Sudoku consiste em checar a existência de números
repetidos em todas as linhas, colunas e regiões. Para checar uma linha,
por exemplo, utilizamos um valor inteiro auxiliar \textit{flag} , onde o bit
\textit{X} de \textit{flag} deve estar no valor lógico um quando o número \textit{X}
está presente naquela linha.
A cada número lido da linha, a sua posição correspondente em \textit{flag} é
marcada atravez de um ou lógico entre a \textit{flag} e 1 deslocado à esquerda
o número incontrado de vezes, de forma a indicar que o número foi visto.
Ao final da linha o valor de \textit{flag} deve ser igual a 0x03FE, ou seja,
 todos os números de 1 à 9 foram encontrados, caso contrario um erro deve ser
apontado naquela linha. O princípio usado para verificar colunas e regiões é
o mesmo.

%%%%%%%%%%%%%%%%%%%%%%%%%%%%%%%%%%%%%%%%%%%%%%%%%%%%%%%%%%%%%%%%%%%%%%%%%%%%%%%%

\section{Abordagem paralela}
A verificação de um Sudoku pode ser vista como a verificação de suas linhas,
colunas e regiões, ou seja, podemos dividir o problema em 27 problemas menores,
checar uma linha, uma coluna ou uma região.
Essa divisão pode ser atingida pelo calculo \textit{load = (SIZE * 3) / nthreads} e
\textit{rest = (SIZE * 3)}, onde cada thread fara no mínimo \textit{load} tarefas,
as \textit{rest} primeiras faram \textit{load} + 1.
\par
Além disso existem 3 tipos de tarefas linhas - tipo 0 -, colunas - tipo 1-  e
regiões - tipo 2-, os tipos são obtidos pelo calculo \textit{job\_type = i / 9}
onde \textit{i} é o número da tarefa. Dentro dos tipos existem 9 tarefas, nas
linhas, por exemplo, exitem 9 colunas, de 0 à 8. O mesmo pode ser observado com
as colunas e regiões. A tarefa pode ser obtida pelo calculo \textit{where = i \% 9}.
\par
De forma que, a tarefa 22, é do tipo 2, ou seja região, e consite em verificar
a 4 região.


%%%%%%%%%%%%%%%%%%%%%%%%%%%%%%%%%%%%%%%%%%%%%%%%%%%%%%%%%%%%%%%%%%%%%%%%%%%%%%%%

\section{Uso}

A a verificação do Sudoku inserido pelo usuário pode ser feito pela linha de comando:
\begin{verbatim}
./sudoku <grid para verificação> <número de threads>
\end{verbatim}

Exemplo de execução correta do programa:
\begin{verbatim}

./sudoku input_grid_correto.txt 4
Quebra-cabecas fornecido:
5 3 4 6 7 8 9 1 2
6 7 2 1 9 5 3 4 8
1 9 8 3 4 2 5 6 7
8 5 9 7 6 1 4 2 3
4 2 6 8 5 3 7 9 1
7 1 3 9 2 4 8 5 6
9 6 1 5 3 7 2 8 4
2 8 7 4 1 9 6 3 5
3 4 5 2 8 6 1 7 9
Erros encontrados: 0
\end{verbatim}

Exemplo de execução incorreta do programa:
\begin{verbatim}
./sudoku input_grid_errado.txt 4
Quebra-cabecas fornecido:
3 5 4 6 7 8 9 1 2
6 7 2 1 9 5 3 4 8
1 9 8 3 4 2 5 6 7
8 5 9 7 6 1 4 2 7
4 2 6 8 5 3 7 9 1
7 1 3 9 2 4 8 5 6
9 6 1 5 3 7 2 8 4
2 8 7 4 1 9 6 3 5
3 4 5 2 8 6 1 7 8
Thread 1: Erro na linha 4
Thread 2: Erro na linha 9
Thread 2: Erro na coluna 1
Thread 2: Erro na coluna 2
Thread 4: Erro na regiao 6
Thread 4: Erro na regiao 9
Thread 3: Erro na coluna 9
Erros encontrados: 7
\end{verbatim}

É possível perceber que, no primeiro caso a solução fornecida não infringe
nenhuma das regra do sudoku. Porém, no segundo caso, temos 7 erros, 3 em
colunas, 2 em linhas e 2 em regiões da solução dada.

\section{Considerações Finais}
Dado que o problema não é escalavel, ou seja, possui um tamanho fixo, e é um
problema pequeno, o \textit{overhead} na utilização de mais threads que o
necessario é substancial, e para esse problema o número de threads "necessárias"
é bem pequeno,  de forma que a comparação de tempo entra a solução sequencial e
paralela não obtem resultados muito satisfatorios.
%%%%%%%%%%%%%%%%%%%%%%%%%%%%%%%%%%%%%%%%%%%%%%%%%%%%%%%%%%%%%%%%%%%%%%%%%%%%%%%%
\end{document}
